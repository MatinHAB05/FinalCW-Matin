\documentclass[12pt]{article}

\usepackage[utf8]{inputenc}
\usepackage{geometry}
\usepackage{listings}
\geometry{a4paper, margin=1in}
\usepackage{graphicx}
\usepackage{hyperref}
\usepackage{fancyhdr}

\pagestyle{fancy}
\fancyhead[L]{\thepage}
\fancyfoot[C]{\texttt{HW-402521189} }

\begin{document}
\title{Computer Workshop\\Final Assignment} 
\date{ 7,Bahman,1402 } %mishof \today zad vali hkob miladi gharar midad
\author{\\ Dr. MalekiMajd \\ \\ \\ \emph{Creator :}\\ \emph{\texttt{Matin Hasanali Baki }}  \\ \\ \underline{SID :    402521189} \\ \\ \\  }
\maketitle

\newpage
\begin{quote}
\end{quote}
\tableofcontents

\newpage




\section*{Information}
links and ...


\section{Git and GitHub}
\subsection{Repository Initialization and Commits}


\subsection{GitHub Actions for LaTeX Compilation}



\section{Exploration Tasks}
\subsection{Vim Advanced Features}
\begin{itemize}
  \item ctrl+n   or  ctrl+p\newline With its help, you can auto complete, for example, the following text:\newline Matin Modem Salam Saeed Return,By typing {M} and then pressing the above command, you can select one of the typed words that starts with this structure, that is, in this example:\newline
  Matin/Modem\newline
  With its help, you can auto complete, for example, the following text:\newline
Matin Modem Salam Saeed Return
By typing {M} and then pressing the above command, you can select one of the typed words that starts with this structure, that is, in this example:
Matin/Modem
  \item gf/gx
      \begin{itemize}
        \item gf
        With its help, you can use the address of a typed directory and create a new buffer file.\newline
        C://Users/Desktop/matin.txt + {gf}
        \item gx
        With its help, you can open a typed website address\newline
        https://quera.org/course/15588/ + {gx}
      \end{itemize}



  \item :X\newline
  With its help, a file can be encrypted and its content cannot be viewed outside Vim, for example, with the help of content extraction commands, or even
  To open that file, you must enter the password that has been set\newline
  :X    ----   Enter encryption key : ****   ----  Enter same encryption key again : ****
\end{itemize}

\subsection{Memory profiling}
This definition occurs when the programmer allocates a memory but does not use it (for any reason, such as forgetfulness, mistakes, incompleteness, or even laziness) and obviously This memory will be involved until it disappears, and the point is that we may no longer need what we allocated, and this means reducing the amount of available memory while running the program.And that means slowing down the execution of the program or even exiting the execution of the program (yes)
\newline reasons :
\begin{itemize}
  \item Not calling for free
  \item Jumping (losing) the address where you allocated memory (and this means that even if we want to delete, we don't know where we saved the memory to delete it, example code:)\newline
  \begin{lstlisting}
  int *ptr=(int * )malloc(sizeof(int)) ;
  int x=5 ;
  ptr=&x ;
  free(????) ;
  \end{lstlisting}
  \item We do not put the conditioning of the fairy and it in the general state (for example:)\newline
  \begin{lstlisting}
    int *ptr=(int * )malloc(sinzeof(int)) ;
    int x ;
    scanf("%d",&x) ;
    if(x%2==0){printf("E") ; free(ptr) ;}
    if(x%2==1){printf("O") ; }
  \end{lstlisting}
  \item If you exit or exit the program while the program is running and the Fairy part has not yet run, it is possible that the memory will not be freed and it will produce a story in the next run and it will become unstable.


\end{itemize}






\subsubsection{Memory Leak}


\subsubsection{Memory profilers}


\subsection{GNU/Linux Bash Scripting}


\subsubsection{fzf}



\subsubsection{Using fzf to find your favorite PDF}



\subsubsection{Opening the file using Zathura}

\section{Git and FOSS}
\subsubsection{README.md}

\subsection{Issues}



\end{document}